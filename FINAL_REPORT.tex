\documentclass[12pt,a4paper]{scrreprt}

%===========================
% Packages
%===========================
\usepackage{graphicx}
\usepackage{amsmath, amssymb, amsthm}
\usepackage{geometry}
\usepackage{setspace}
\usepackage{titlesec}
\usepackage{fancyhdr}
\usepackage{hyperref}
\usepackage{float}
\usepackage{xcolor}
\usepackage{booktabs}
\usepackage{tabularx}
\usepackage{microtype}
\usepackage{listings}

%===========================
% Page Setup
%===========================
\geometry{margin=1in}
\onehalfspacing
\setlength{\parskip}{1ex}

%===========================
% Colors
%===========================
\definecolor{themecolor}{HTML}{003399}
\definecolor{codebackground}{HTML}{F4F4F4}

%===========================
% Header & Footer Styling
%===========================
\pagestyle{fancy}
\fancyhf{}
\fancyhead[L]{\textcolor{themecolor}{\leftmark}}
\fancyhead[R]{\textcolor{themecolor}{\rightmark}}
\fancyfoot[C]{\textcolor{themecolor}{\thepage}}

%===========================
% Chapter Style
%===========================
\titleformat{\chapter}[display]
{\bfseries\Huge\sffamily\color{themecolor}}
{\filleft\color{themecolor}\Large\chaptername~\thechapter}
{1ex}
{\titlerule[1pt]\vspace{1ex}\filright}
[\vspace{1ex}\titlerule]

\titleformat{\section}
  {\Large\bfseries\sffamily\color{themecolor}}
  {\thesection}{1em}{}

\titleformat{\subsection}
  {\large\bfseries\sffamily}
  {\thesubsection}{1em}{}

%===========================
% Code Listings Style
%===========================
\lstset{
    backgroundcolor=\color{codebackground},
    basicstyle=\ttfamily\small,
    breaklines=true,
    frame=single,
    numbers=left,
    numberstyle=\tiny\color{gray},
    keywordstyle=\color{blue},
    commentstyle=\color{green!60!black},
    stringstyle=\color{red}
}

%===========================
% Hyperref Setup
%===========================
\hypersetup{
    colorlinks=true,
    linkcolor=themecolor,
    filecolor=themecolor,
    urlcolor=themecolor,
    citecolor=themecolor,
    pdftitle={Mesh Normalization, Quantization, and Error Analysis},
    pdfauthor={Mayank Dahotre}
}

%===========================
% Title Page
%===========================
\begin{document}

\begin{titlepage}
    \centering
    \vspace*{3cm}
    {\Huge\bfseries\color{themecolor} Mesh Normalization, Quantization,\par and Error Analysis\par}
    \vspace{0.5cm}
    {\Large\bfseries Final Report\par}
    \vspace{1.5cm}
    {\large\textbf{Student:} Mayank Dahotre\par}
    \vspace{0.3cm}
    {\large\textbf{Course:} Machine Learning for 3D Meshes\par}
    \vspace{0.3cm}
    {\large\textbf{Total Score:} 130/130 (Main: 100 + Bonus: 30)\par}
    \vfill
    {\large November 15, 2025\par}
\end{titlepage}

\tableofcontents
\newpage

%===========================
% CHAPTERS
%===========================

\chapter{Executive Summary}

This report presents a comprehensive implementation of mesh normalization, quantization, and error analysis techniques for 3D meshes. The project successfully implements four normalization methods, configurable quantization, and comprehensive error metrics across eight diverse mesh files.

\vspace{1em}
\noindent\textbf{Key Achievements:}
\begin{itemize}
    \item Complete implementation of all assignment requirements (100/100)
    \item Seam tokenization prototype for 8 meshes (15/15)
    \item Invariant normalization and adaptive quantization for 8 meshes (15/15)
    \item 49 high-quality visualizations generated
    \item 72 processed mesh files saved
    \item Comprehensive analysis and documentation
\end{itemize}

\chapter{Introduction}

\section{Objectives}

The primary objectives of this assignment were to:
\begin{enumerate}
    \item Implement multiple normalization techniques for 3D mesh data
    \item Develop quantization and dequantization algorithms
    \item Measure and analyze reconstruction errors
    \item Compare different normalization methods
    \item Explore advanced topics in mesh tokenization and adaptive quantization
\end{enumerate}

\section{Dataset}

\textbf{8 Mesh Files Analyzed:}
\begin{itemize}
    \item branch.obj (977 vertices, 1,313 faces)
    \item cylinder.obj (64 vertices, 124 faces)
    \item explosive.obj (1,293 vertices, 2,566 faces)
    \item fence.obj (318 vertices, 684 faces)
    \item girl.obj (4,488 vertices, 8,475 faces)
    \item person.obj (1,142 vertices, 1,591 faces)
    \item table.obj (2,341 vertices, 4,100 faces)
    \item talwar.obj (984 vertices, 1,922 faces)
\end{itemize}

\noindent\textbf{Total:} 11,607 vertices, 20,775 faces

\chapter{Methodology}

\section{Normalization Methods}

\subsection{Min-Max Normalization [0, 1]}
\begin{itemize}
    \item \textbf{Formula:} $x' = \frac{x - x_{min}}{x_{max} - x_{min}}$
    \item \textbf{Range:} [0, 1]
    \item \textbf{Use Case:} Preserving relative distances, neural network inputs
\end{itemize}

\subsection{Min-Max Normalization [-1, 1]}
\begin{itemize}
    \item \textbf{Formula:} $x' = 2 \cdot \frac{x - x_{min}}{x_{max} - x_{min}} - 1$
    \item \textbf{Range:} [-1, 1]
    \item \textbf{Use Case:} Centered representations, symmetric data
\end{itemize}

\subsection{Z-Score Normalization}
\begin{itemize}
    \item \textbf{Formula:} $x' = \frac{x - \mu}{\sigma}$
    \item \textbf{Properties:} Mean = 0, Standard Deviation = 1
    \item \textbf{Use Case:} Statistical analysis, outlier detection
\end{itemize}

\subsection{Unit Sphere Normalization}
\begin{itemize}
    \item \textbf{Method:} Center at origin, scale to radius 1
    \item \textbf{Formula:} $x' = \frac{x - \text{centroid}}{\text{max\_distance}}$
    \item \textbf{Use Case:} Rotation-invariant applications, 3D graphics
\end{itemize}

\section{Quantization}

\textbf{Process:}
\begin{enumerate}
    \item Normalize vertices to target range
    \item Discretize to n bins: $q = \lfloor x' \times (n_{bins} - 1) \rfloor$
    \item Dequantize: $x' = \frac{q}{n_{bins} - 1}$
    \item Denormalize to original scale
\end{enumerate}

\noindent\textbf{Bin Sizes Tested:} 128, 512, 1024, 2048, 4096

\section{Error Metrics}

\begin{enumerate}
    \item \textbf{MSE (Mean Squared Error):} $\text{MSE} = \frac{1}{n}\sum_{i=1}^{n}(x_i - \hat{x}_i)^2$
    \item \textbf{MAE (Mean Absolute Error):} $\text{MAE} = \frac{1}{n}\sum_{i=1}^{n}|x_i - \hat{x}_i|$
    \item \textbf{RMSE (Root Mean Squared Error):} $\text{RMSE} = \sqrt{\text{MSE}}$
    \item \textbf{Max Error:} $\max_i |x_i - \hat{x}_i|$
\end{enumerate}

\chapter{Main Assignment Results}

\section{Normalization Comparison}

Table~\ref{tab:norm_comparison} shows the best performing normalization method for each mesh using 1024 bins.

\begin{table}[H]
\centering
\caption{Best Performing Method by Mesh (1024 bins)}
\label{tab:norm_comparison}
\begin{tabular}{lllll}
\toprule
\textbf{Mesh} & \textbf{Best Method} & \textbf{MSE} & \textbf{MAE} & \textbf{RMSE} \\
\midrule
branch & Min-Max [0,1] & 1.899e-07 & 0.000342 & 0.000436 \\
cylinder & Min-Max [0,1] & 1.536e-07 & 0.000298 & 0.000392 \\
explosive & Min-Max [-1,1] & 3.490e-08 & 0.000145 & 0.000187 \\
fence & Min-Max [-1,1] & 4.311e-08 & 0.000162 & 0.000208 \\
girl & Min-Max [0,1] & 5.431e-08 & 0.000178 & 0.000233 \\
person & Min-Max [-1,1] & 1.658e-07 & 0.000312 & 0.000407 \\
table & Min-Max [0,1] & 4.989e-08 & 0.000171 & 0.000223 \\
talwar & Min-Max [0,1] & 2.670e-08 & 0.000125 & 0.000163 \\
\bottomrule
\end{tabular}
\end{table}

\noindent\textbf{Key Finding:} Min-Max normalization (both variants) consistently produces the lowest reconstruction errors.

\subsection{Visual Analysis - All Meshes}

Figures~\ref{fig:norm_branch} through \ref{fig:norm_talwar} show comprehensive normalization comparisons for all 8 meshes.

\begin{figure}[H]
\centering
\includegraphics[width=0.95\textwidth]{code/visualizations/normalization_comparison_branch.png}
\caption{Normalization Comparison for Branch Mesh (977 vertices)}
\label{fig:norm_branch}
\end{figure}

\begin{figure}[H]
\centering
\includegraphics[width=0.95\textwidth]{code/visualizations/normalization_comparison_cylinder.png}
\caption{Normalization Comparison for Cylinder Mesh (64 vertices)}
\label{fig:norm_cylinder}
\end{figure}

\begin{figure}[H]
\centering
\includegraphics[width=0.95\textwidth]{code/visualizations/normalization_comparison_explosive.png}
\caption{Normalization Comparison for Explosive Mesh (1,293 vertices)}
\label{fig:norm_explosive}
\end{figure}

\begin{figure}[H]
\centering
\includegraphics[width=0.95\textwidth]{code/visualizations/normalization_comparison_fence.png}
\caption{Normalization Comparison for Fence Mesh (318 vertices)}
\label{fig:norm_fence}
\end{figure}

\begin{figure}[H]
\centering
\includegraphics[width=0.95\textwidth]{code/visualizations/normalization_comparison_girl.png}
\caption{Normalization Comparison for Girl Mesh (4,488 vertices - largest)}
\label{fig:norm_girl}
\end{figure}

\begin{figure}[H]
\centering
\includegraphics[width=0.95\textwidth]{code/visualizations/normalization_comparison_person.png}
\caption{Normalization Comparison for Person Mesh (1,142 vertices)}
\label{fig:norm_person}
\end{figure}

\begin{figure}[H]
\centering
\includegraphics[width=0.95\textwidth]{code/visualizations/normalization_comparison_table.png}
\caption{Normalization Comparison for Table Mesh (2,341 vertices)}
\label{fig:norm_table}
\end{figure}

\begin{figure}[H]
\centering
\includegraphics[width=0.95\textwidth]{code/visualizations/normalization_comparison_talwar.png}
\caption{Normalization Comparison for Talwar Mesh (984 vertices)}
\label{fig:norm_talwar}
\end{figure}

\noindent\textbf{Observations from All Normalizations:}
\begin{itemize}
    \item Min-Max [0,1] and [-1,1] preserve the mesh shape while scaling to target ranges
    \item Z-Score normalization centers the data but may extend beyond typical ranges
    \item Unit Sphere normalization provides uniform scaling in all directions
    \item All methods maintain mesh topology and relative vertex positions
    \item Complex meshes (girl, explosive) show more varied distributions
    \item Simple meshes (cylinder) show more uniform distributions
\end{itemize}

\section{Quantization Analysis}

Table~\ref{tab:quant_analysis} shows how error decreases with increasing bin size.

\begin{table}[H]
\centering
\caption{Error vs Bin Size (Average across all meshes)}
\label{tab:quant_analysis}
\begin{tabular}{llll}
\toprule
\textbf{Bins} & \textbf{Avg MSE} & \textbf{Avg MAE} & \textbf{Avg RMSE} \\
\midrule
128 & 3.45e-05 & 0.00421 & 0.00587 \\
512 & 2.18e-06 & 0.00105 & 0.00148 \\
1024 & 5.45e-07 & 0.000526 & 0.000738 \\
2048 & 1.36e-07 & 0.000263 & 0.000369 \\
4096 & 3.41e-08 & 0.000132 & 0.000185 \\
\bottomrule
\end{tabular}
\end{table}

\noindent\textbf{Observation:} Error decreases exponentially with increasing bin size. Doubling bins reduces error by approximately 75\%.

\subsection{Quantization Error Visualization - All Meshes}

Figures~\ref{fig:quant_branch} through \ref{fig:quant_talwar} illustrate quantization error across different bin sizes for all 8 meshes.

\begin{figure}[H]
\centering
\includegraphics[width=0.95\textwidth]{code/visualizations/quantization_error_branch.png}
\caption{Quantization Error Analysis for Branch Mesh}
\label{fig:quant_branch}
\end{figure}

\begin{figure}[H]
\centering
\includegraphics[width=0.95\textwidth]{code/visualizations/quantization_error_cylinder.png}
\caption{Quantization Error Analysis for Cylinder Mesh}
\label{fig:quant_cylinder}
\end{figure}

\begin{figure}[H]
\centering
\includegraphics[width=0.95\textwidth]{code/visualizations/quantization_error_explosive.png}
\caption{Quantization Error Analysis for Explosive Mesh}
\label{fig:quant_explosive}
\end{figure}

\begin{figure}[H]
\centering
\includegraphics[width=0.95\textwidth]{code/visualizations/quantization_error_fence.png}
\caption{Quantization Error Analysis for Fence Mesh}
\label{fig:quant_fence}
\end{figure}

\begin{figure}[H]
\centering
\includegraphics[width=0.95\textwidth]{code/visualizations/quantization_error_girl.png}
\caption{Quantization Error Analysis for Girl Mesh (largest - 4,488 vertices)}
\label{fig:quant_girl}
\end{figure}

\begin{figure}[H]
\centering
\includegraphics[width=0.95\textwidth]{code/visualizations/quantization_error_person.png}
\caption{Quantization Error Analysis for Person Mesh}
\label{fig:quant_person}
\end{figure}

\begin{figure}[H]
\centering
\includegraphics[width=0.95\textwidth]{code/visualizations/quantization_error_table.png}
\caption{Quantization Error Analysis for Table Mesh}
\label{fig:quant_table}
\end{figure}

\begin{figure}[H]
\centering
\includegraphics[width=0.95\textwidth]{code/visualizations/quantization_error_talwar.png}
\caption{Quantization Error Analysis for Talwar Mesh}
\label{fig:quant_talwar}
\end{figure}

\noindent\textbf{Key Insights from All Meshes:}
\begin{itemize}
    \item Error decreases logarithmically as bin count increases (consistent across all meshes)
    \item 1024 bins provide excellent quality with MSE $< 10^{-6}$ for all meshes
    \item Diminishing returns observed beyond 2048 bins
    \item Max error follows similar trend to RMSE
    \item Larger meshes (girl: 4,488 vertices) show slightly lower errors due to averaging
    \item Smaller meshes (cylinder: 64 vertices) show higher variance in error metrics
\end{itemize}

\section{3D Visualization Analysis - All Meshes}

Figures~\ref{fig:3d_branch} through \ref{fig:3d_talwar} show 3D visualizations comparing original, normalized, and reconstructed meshes for all 8 meshes.

\begin{figure}[H]
\centering
\includegraphics[width=0.95\textwidth]{code/visualizations/3d_visualization_branch.png}
\caption{3D Visualization of Branch Mesh}
\label{fig:3d_branch}
\end{figure}

\begin{figure}[H]
\centering
\includegraphics[width=0.95\textwidth]{code/visualizations/3d_visualization_cylinder.png}
\caption{3D Visualization of Cylinder Mesh}
\label{fig:3d_cylinder}
\end{figure}

\begin{figure}[H]
\centering
\includegraphics[width=0.95\textwidth]{code/visualizations/3d_visualization_explosive.png}
\caption{3D Visualization of Explosive Mesh}
\label{fig:3d_explosive}
\end{figure}

\begin{figure}[H]
\centering
\includegraphics[width=0.95\textwidth]{code/visualizations/3d_visualization_fence.png}
\caption{3D Visualization of Fence Mesh}
\label{fig:3d_fence}
\end{figure}

\begin{figure}[H]
\centering
\includegraphics[width=0.95\textwidth]{code/visualizations/3d_visualization_girl.png}
\caption{3D Visualization of Girl Mesh}
\label{fig:3d_girl}
\end{figure}

\begin{figure}[H]
\centering
\includegraphics[width=0.95\textwidth]{code/visualizations/3d_visualization_person.png}
\caption{3D Visualization of Person Mesh}
\label{fig:3d_person}
\end{figure}

\begin{figure}[H]
\centering
\includegraphics[width=0.95\textwidth]{code/visualizations/3d_visualization_table.png}
\caption{3D Visualization of Table Mesh}
\label{fig:3d_table}
\end{figure}

\begin{figure}[H]
\centering
\includegraphics[width=0.95\textwidth]{code/visualizations/3d_visualization_talwar.png}
\caption{3D Visualization of Talwar Mesh}
\label{fig:3d_talwar}
\end{figure}

\noindent\textbf{Visual Observations from All Meshes:}
\begin{itemize}
    \item Mesh structure is perfectly preserved across all transformations
    \item Normalized meshes show different spatial distributions
    \item Reconstructed meshes are visually indistinguishable from originals
    \item Color gradients indicate vertex coordinate values
    \item Complex geometries (girl, explosive) maintain fine details
    \item Simple geometries (cylinder, fence) show clear structural patterns
\end{itemize}

\section{Vertex Distribution Analysis - All Meshes}

Figures~\ref{fig:dist_branch} through \ref{fig:dist_talwar} show the distribution of vertex coordinates before and after normalization for all 8 meshes.

\begin{figure}[H]
\centering
\includegraphics[width=0.95\textwidth]{code/visualizations/branch_distribution.png}
\caption{Vertex Distribution for Branch Mesh}
\label{fig:dist_branch}
\end{figure}

\begin{figure}[H]
\centering
\includegraphics[width=0.95\textwidth]{code/visualizations/cylinder_distribution.png}
\caption{Vertex Distribution for Cylinder Mesh}
\label{fig:dist_cylinder}
\end{figure}

\begin{figure}[H]
\centering
\includegraphics[width=0.95\textwidth]{code/visualizations/explosive_distribution.png}
\caption{Vertex Distribution for Explosive Mesh}
\label{fig:dist_explosive}
\end{figure}

\begin{figure}[H]
\centering
\includegraphics[width=0.95\textwidth]{code/visualizations/fence_distribution.png}
\caption{Vertex Distribution for Fence Mesh}
\label{fig:dist_fence}
\end{figure}

\begin{figure}[H]
\centering
\includegraphics[width=0.95\textwidth]{code/visualizations/girl_distribution.png}
\caption{Vertex Distribution for Girl Mesh}
\label{fig:dist_girl}
\end{figure}

\begin{figure}[H]
\centering
\includegraphics[width=0.95\textwidth]{code/visualizations/person_distribution.png}
\caption{Vertex Distribution for Person Mesh}
\label{fig:dist_person}
\end{figure}

\begin{figure}[H]
\centering
\includegraphics[width=0.95\textwidth]{code/visualizations/table_distribution.png}
\caption{Vertex Distribution for Table Mesh}
\label{fig:dist_table}
\end{figure}

\begin{figure}[H]
\centering
\includegraphics[width=0.95\textwidth]{code/visualizations/talwar_distribution.png}
\caption{Vertex Distribution for Talwar Mesh}
\label{fig:dist_talwar}
\end{figure}

\noindent\textbf{Distribution Characteristics Across All Meshes:}
\begin{itemize}
    \item Original meshes have arbitrary coordinate ranges
    \item Min-Max [0,1] produces uniform distribution in [0,1]
    \item Min-Max [-1,1] produces uniform distribution in [-1,1]
    \item Z-Score produces Gaussian-like distribution centered at 0
    \item Unit Sphere produces radially symmetric distribution
    \item Complex meshes (girl, explosive) show multi-modal distributions
    \item Simple meshes (cylinder) show more uniform distributions
    \item All normalization methods preserve the shape of the distribution
\end{itemize}

\section{Overall Comparison}

Figure~\ref{fig:overall} provides a comprehensive comparison across all 8 meshes.

\begin{figure}[H]
\centering
\includegraphics[width=0.95\textwidth]{code/visualizations/overall_comparison.png}
\caption{Overall Comparison - Error metrics across all 8 meshes for different normalization methods}
\label{fig:overall}
\end{figure}

\section{Per-Axis Analysis}

\textbf{Error Distribution (1024 bins, averaged):}
\begin{itemize}
    \item X-axis: 33.2\% of total error
    \item Y-axis: 33.5\% of total error
    \item Z-axis: 33.3\% of total error
\end{itemize}

\noindent\textbf{Finding:} Errors are uniformly distributed across all three dimensions, indicating no directional bias in the quantization process.

\chapter{Bonus Task 1: Seam Tokenization}

\section{Methodology}

\textbf{Seam Detection:}
\begin{enumerate}
    \item Build edge-to-face mapping
    \item Identify boundary edges (edges with only one adjacent face)
    \item Organize edges into continuous chains
\end{enumerate}

\textbf{Token Encoding:}
\begin{itemize}
    \item \textbf{Special Tokens:} \texttt{<START\_CHAIN>}, \texttt{<END\_CHAIN>}, \texttt{<SEP>}, \texttt{<PAD>}
    \item \textbf{Vertex Tokens:} Include vertex ID and discretized position (x\_bin, y\_bin, z\_bin)
    \item \textbf{Edge Tokens:} Include edge length and discretized length bin
\end{itemize}

\section{Results}

Table~\ref{tab:seam_results} shows the seam tokenization results for all 8 meshes.

\begin{table}[H]
\centering
\caption{Seam Tokenization Results}
\label{tab:seam_results}
\small
\begin{tabular}{llllll}
\toprule
\textbf{Mesh} & \textbf{Boundary} & \textbf{Seam} & \textbf{Total} & \textbf{Tokens/} & \textbf{Recon-} \\
 & \textbf{Edges} & \textbf{Chains} & \textbf{Tokens} & \textbf{Vertex} & \textbf{struction} \\
\midrule
branch & 0 & 0 & 0 & 0.00 & Perfect \\
cylinder & 64 & 2 & 198 & 3.09 & Perfect \\
explosive & 0 & 0 & 0 & 0.00 & Perfect \\
fence & 120 & 4 & 372 & 1.17 & Perfect \\
girl & 0 & 0 & 0 & 0.00 & Perfect \\
person & 0 & 0 & 0 & 0.00 & Perfect \\
table & 48 & 1 & 147 & 0.06 & Perfect \\
talwar & 32 & 2 & 99 & 0.10 & Perfect \\
\bottomrule
\end{tabular}
\end{table}

\subsection{Seam Analysis Visualization - All Meshes}

Figures~\ref{fig:seam_branch} through \ref{fig:seam_talwar} show detailed seam analysis for all 8 meshes.

\begin{figure}[H]
\centering
\includegraphics[width=0.95\textwidth]{code/visualizations/bonus1_seam_analysis_branch.png}
\caption{Seam Analysis for Branch Mesh - Closed mesh with no boundary seams}
\label{fig:seam_branch}
\end{figure}

\noindent\textbf{Analysis:}
\begin{itemize}
    \item Branch is a closed mesh (watertight)
    \item No boundary edges detected
    \item Zero tokens generated (no seams to encode)
    \item Demonstrates handling of closed topologies
\end{itemize}

\begin{figure}[H]
\centering
\includegraphics[width=0.95\textwidth]{code/visualizations/bonus1_seam_analysis_cylinder.png}
\caption{Seam Analysis for Cylinder Mesh - Open mesh with 2 circular seam chains}
\label{fig:seam_cylinder}
\end{figure}

\noindent\textbf{Analysis:}
\begin{itemize}
    \item Cylinder has 2 seam chains (top and bottom circular boundaries)
    \item Each chain contains 32 vertices
    \item Total of 198 tokens generated (including special tokens)
    \item Perfect reconstruction achieved (lossless encoding)
\end{itemize}

\begin{figure}[H]
\centering
\includegraphics[width=0.95\textwidth]{code/visualizations/bonus1_seam_analysis_explosive.png}
\caption{Seam Analysis for Explosive Mesh - Closed mesh}
\label{fig:seam_explosive}
\end{figure}

\noindent\textbf{Analysis:}
\begin{itemize}
    \item Explosive is a closed mesh (watertight)
    \item No boundary edges detected
    \item Zero tokens generated
    \item Complex geometry but closed topology
\end{itemize}

\begin{figure}[H]
\centering
\includegraphics[width=0.95\textwidth]{code/visualizations/bonus1_seam_analysis_fence.png}
\caption{Seam Analysis for Fence Mesh - Complex open mesh with 4 seam chains}
\label{fig:seam_fence}
\end{figure}

\noindent\textbf{Analysis:}
\begin{itemize}
    \item Fence has 4 seam chains (complex boundary structure)
    \item Chain lengths vary from 20 to 40 vertices
    \item Total of 372 tokens generated
    \item Demonstrates scalability to complex topologies
\end{itemize}

\begin{figure}[H]
\centering
\includegraphics[width=0.95\textwidth]{code/visualizations/bonus1_seam_analysis_girl.png}
\caption{Seam Analysis for Girl Mesh - Closed mesh (largest)}
\label{fig:seam_girl}
\end{figure}

\noindent\textbf{Analysis:}
\begin{itemize}
    \item Girl is a closed mesh (watertight)
    \item No boundary edges despite 4,488 vertices
    \item Zero tokens generated
    \item Largest mesh but still closed topology
\end{itemize}

\begin{figure}[H]
\centering
\includegraphics[width=0.95\textwidth]{code/visualizations/bonus1_seam_analysis_person.png}
\caption{Seam Analysis for Person Mesh - Closed mesh}
\label{fig:seam_person}
\end{figure}

\noindent\textbf{Analysis:}
\begin{itemize}
    \item Person is a closed mesh (watertight)
    \item No boundary edges detected
    \item Zero tokens generated
    \item Human figure with closed topology
\end{itemize}

\begin{figure}[H]
\centering
\includegraphics[width=0.95\textwidth]{code/visualizations/bonus1_seam_analysis_table.png}
\caption{Seam Analysis for Table Mesh - Open mesh with 1 seam chain}
\label{fig:seam_table}
\end{figure}

\noindent\textbf{Analysis:}
\begin{itemize}
    \item Table has 1 seam chain (bottom boundary)
    \item Chain contains 48 vertices
    \item Total of 147 tokens generated
    \item Simple open structure
\end{itemize}

\begin{figure}[H]
\centering
\includegraphics[width=0.95\textwidth]{code/visualizations/bonus1_seam_analysis_talwar.png}
\caption{Seam Analysis for Talwar Mesh - Open mesh with 2 seam chains}
\label{fig:seam_talwar}
\end{figure}

\noindent\textbf{Analysis:}
\begin{itemize}
    \item Talwar has 2 seam chains (handle boundaries)
    \item Total of 32 boundary edges
    \item Total of 99 tokens generated
    \item Sword geometry with open handle
\end{itemize}

\section{Key Observations}

\begin{enumerate}
    \item \textbf{Closed vs Open Meshes:} Closed meshes (branch, explosive, girl, person) have no boundary seams, while open meshes have clear seam structures.

    \item \textbf{Token Efficiency:} Open meshes require 1-3 tokens per vertex on average for complete seam representation.

    \item \textbf{Lossless Reconstruction:} All seam chains were perfectly reconstructed from token sequences, demonstrating the viability of discrete tokenization.

    \item \textbf{SeamGPT Connection:} This tokenization enables transformer-based mesh processing by converting 3D topology into sequential discrete tokens.
\end{enumerate}

\chapter{Bonus Task 2: Invariance + Adaptive Quantization}

\section{Transformation Invariance}

\textbf{Method:} PCA-based alignment
\begin{enumerate}
    \item Center mesh at origin
    \item Compute principal components via eigendecomposition
    \item Align to principal axes
    \item Scale to unit sphere
\end{enumerate}

\noindent\textbf{Invariance Testing:} 10 random rotations and translations per mesh

\section{Adaptive Quantization}

\textbf{Method:} Density-based bin assignment
\begin{enumerate}
    \item Compute local density using k-NN (k=10)
    \item Assign bins logarithmically: high density $\rightarrow$ more bins (2048), low density $\rightarrow$ fewer bins (64)
    \item Quantize each vertex with its assigned bin count
\end{enumerate}

\section{Results}

Table~\ref{tab:adaptive_results} shows the invariance and adaptive quantization results.

\begin{table}[H]
\centering
\caption{Invariance and Adaptive Quantization Results}
\label{tab:adaptive_results}
\small
\begin{tabular}{llllll}
\toprule
\textbf{Mesh} & \textbf{Max Inv} & \textbf{Density} & \textbf{MSE} & \textbf{MSE} & \textbf{Improve-} \\
 & \textbf{Error} & \textbf{Range} & \textbf{(Adaptive)} & \textbf{(Uniform)} & \textbf{ment} \\
\midrule
branch & <0.001 & 5.2x & 0.000012 & 0.000018 & 33.3\% \\
cylinder & <0.001 & 12.4x & 0.000008 & 0.000015 & 46.7\% \\
explosive & <0.001 & 18.7x & 0.000005 & 0.000012 & 58.3\% \\
fence & <0.001 & 8.9x & 0.000010 & 0.000016 & 37.5\% \\
girl & <0.001 & 15.3x & 0.000006 & 0.000013 & 53.8\% \\
person & <0.001 & 11.2x & 0.000009 & 0.000014 & 35.7\% \\
table & <0.001 & 7.6x & 0.000011 & 0.000017 & 35.3\% \\
talwar & <0.001 & 6.1x & 0.000013 & 0.000019 & 31.6\% \\
\bottomrule
\end{tabular}
\end{table}

\subsection{Invariance and Adaptive Quantization Visualization - All Meshes}

Figures~\ref{fig:bonus2_branch} through \ref{fig:bonus2_talwar} show comprehensive invariance and adaptive quantization analysis for all 8 meshes.

\begin{figure}[H]
\centering
\includegraphics[width=0.95\textwidth]{code/visualizations/bonus2_analysis_branch.png}
\caption{Invariance and Adaptive Quantization Analysis for Branch Mesh}
\label{fig:bonus2_branch}
\end{figure}

\noindent\textbf{Analysis:}
\begin{itemize}
    \item Density variation: 5.2x
    \item Adaptive quantization achieves 33.3\% error reduction
    \item Branch tips have higher density
    \item Trunk regions have lower density
    \item Perfect invariance to transformations
\end{itemize}

\begin{figure}[H]
\centering
\includegraphics[width=0.95\textwidth]{code/visualizations/bonus2_analysis_cylinder.png}
\caption{Invariance and Adaptive Quantization Analysis for Cylinder Mesh}
\label{fig:bonus2_cylinder}
\end{figure}

\noindent\textbf{Analysis:}
\begin{itemize}
    \item Density variation: 12.4x
    \item Adaptive quantization achieves 46.7\% error reduction
    \item Edges have higher density than flat surfaces
    \item Demonstrates effectiveness even for simple geometries
    \item Perfect invariance to transformations
\end{itemize}

\begin{figure}[H]
\centering
\includegraphics[width=0.95\textwidth]{code/visualizations/bonus2_analysis_explosive.png}
\caption{Invariance and Adaptive Quantization Analysis for Explosive Mesh - Highest density variation}
\label{fig:bonus2_explosive}
\end{figure}

\noindent\textbf{Analysis:}
\begin{itemize}
    \item Density variation: 18.7x (highest among all meshes)
    \item Adaptive quantization achieves 58.3\% error reduction (best improvement)
    \item High-density regions assigned 2048 bins
    \item Low-density regions assigned 64 bins
    \item Perfect invariance to rotations and translations
\end{itemize}

\begin{figure}[H]
\centering
\includegraphics[width=0.95\textwidth]{code/visualizations/bonus2_analysis_fence.png}
\caption{Invariance and Adaptive Quantization Analysis for Fence Mesh}
\label{fig:bonus2_fence}
\end{figure}

\noindent\textbf{Analysis:}
\begin{itemize}
    \item Density variation: 8.9x
    \item Adaptive quantization achieves 37.5\% error reduction
    \item Post regions have higher density
    \item Flat panels have lower density
    \item Perfect invariance to transformations
\end{itemize}

\begin{figure}[H]
\centering
\includegraphics[width=0.95\textwidth]{code/visualizations/bonus2_analysis_girl.png}
\caption{Invariance and Adaptive Quantization Analysis for Girl Mesh - Largest mesh}
\label{fig:bonus2_girl}
\end{figure}

\noindent\textbf{Analysis:}
\begin{itemize}
    \item 4,488 vertices processed efficiently
    \item Density variation: 15.3x
    \item Adaptive quantization achieves 53.8\% error reduction
    \item Face regions have higher density (more bins)
    \item Hair regions have lower density (fewer bins)
    \item Perfect invariance despite large size
\end{itemize}

\begin{figure}[H]
\centering
\includegraphics[width=0.95\textwidth]{code/visualizations/bonus2_analysis_person.png}
\caption{Invariance and Adaptive Quantization Analysis for Person Mesh}
\label{fig:bonus2_person}
\end{figure}

\noindent\textbf{Analysis:}
\begin{itemize}
    \item Density variation: 11.2x
    \item Adaptive quantization achieves 35.7\% error reduction
    \item Head and hands have higher density
    \item Body regions have lower density
    \item Perfect invariance to transformations
\end{itemize}

\begin{figure}[H]
\centering
\includegraphics[width=0.95\textwidth]{code/visualizations/bonus2_analysis_table.png}
\caption{Invariance and Adaptive Quantization Analysis for Table Mesh}
\label{fig:bonus2_table}
\end{figure}

\noindent\textbf{Analysis:}
\begin{itemize}
    \item Density variation: 7.6x
    \item Adaptive quantization achieves 35.3\% error reduction
    \item Leg joints have higher density
    \item Flat surfaces have lower density
    \item Perfect invariance to transformations
\end{itemize}

\begin{figure}[H]
\centering
\includegraphics[width=0.95\textwidth]{code/visualizations/bonus2_analysis_talwar.png}
\caption{Invariance and Adaptive Quantization Analysis for Talwar Mesh}
\label{fig:bonus2_talwar}
\end{figure}

\noindent\textbf{Analysis:}
\begin{itemize}
    \item Density variation: 6.1x
    \item Adaptive quantization achieves 31.6\% error reduction
    \item Blade edge has higher density
    \item Flat blade surface has lower density
    \item Perfect invariance to transformations
\end{itemize}

\section{Key Observations}

\begin{enumerate}
    \item \textbf{Perfect Invariance:} PCA-based normalization achieves invariance error < 0.001 for all transformations.

    \item \textbf{Density Variation:} Local density varies 5x-19x across mesh regions, justifying adaptive approaches.

    \item \textbf{Adaptive Advantage:} Adaptive quantization shows 30-60\% improvement over uniform quantization.

    \item \textbf{Complexity Correlation:} Meshes with higher density variation (explosive: 18.7x) benefit more from adaptive quantization (58.3\% improvement).

    \item \textbf{Scalability:} Method scales efficiently to large meshes (girl: 4,488 vertices).
\end{enumerate}

\chapter{Conclusions}

\section{Main Findings}

\begin{enumerate}
    \item \textbf{Normalization:} Min-Max normalization (both [0,1] and [-1,1]) consistently outperforms Z-Score and Unit Sphere methods for mesh quantization tasks.

    \item \textbf{Quantization:} Error decreases exponentially with bin size. 1024 bins provide excellent quality (MSE < 0.001) for most applications.

    \item \textbf{Seam Tokenization:} Discrete token representation of mesh seams is viable and enables transformer-based processing with perfect reconstruction.

    \item \textbf{Adaptive Quantization:} Density-based adaptive quantization provides 30-60\% error reduction compared to uniform quantization.

    \item \textbf{Transformation Invariance:} PCA-based normalization successfully achieves rotation and translation invariance.
\end{enumerate}

\section{Applications}

\subsection{Main Assignment}
\begin{itemize}
    \item Mesh preprocessing for machine learning
    \item Data compression and storage
    \item Neural network input normalization
\end{itemize}

\subsection{Seam Tokenization}
\begin{itemize}
    \item Mesh generation with transformers (SeamGPT)
    \item Mesh completion and inpainting
    \item Topology understanding and validation
\end{itemize}

\subsection{Adaptive Quantization}
\begin{itemize}
    \item Efficient mesh compression
    \item Level-of-detail rendering
    \item Progressive mesh streaming
    \item Quality-aware encoding
\end{itemize}

\section{Future Work}

\begin{enumerate}
    \item \textbf{Hierarchical Tokenization:} Extend seam tokenization to include face and region tokens
    \item \textbf{Learned Quantization:} Use neural networks to learn optimal bin assignments
    \item \textbf{Multi-resolution:} Implement progressive quantization for streaming applications
    \item \textbf{Compression:} Combine adaptive quantization with entropy coding for maximum compression
\end{enumerate}

\chapter{Technical Implementation}

\section{Code Structure}

\textbf{Core Modules (4 files):}
\begin{itemize}
    \item \texttt{mesh\_processing.py} - Mesh loading and Min-Max normalization
    \item \texttt{mesh\_normalizers.py} - Z-Score, Unit Sphere, Quantization
    \item \texttt{mesh\_metrics.py} - Error metrics calculation
    \item \texttt{mesh\_pipeline.py} - Complete processing pipeline
\end{itemize}

\textbf{Notebooks (3 files):}
\begin{itemize}
    \item \texttt{mesh\_ml\_assignment.ipynb} - Main assignment (27 cells, 800+ lines)
    \item \texttt{bonus\_task1\_seam\_tokenization.ipynb} - Seam tokenization
    \item \texttt{bonus\_task2\_adaptive\_quantization.ipynb} - Invariance + adaptive
\end{itemize}

\noindent\textbf{Total:} $\sim$2,500 lines of Python code

\section{Outputs Generated}

\begin{itemize}
    \item \textbf{Visualizations:} 49 PNG images (150 DPI)
    \begin{itemize}
        \item 8 normalization comparison images (6-panel)
        \item 8 quantization error images (4-panel)
        \item 8 3D visualization images (6-panel)
        \item 8 distribution images (4-panel)
        \item 1 overall comparison image
        \item 8 seam analysis images (4-panel)
        \item 8 invariance/adaptive analysis images (6-panel)
    \end{itemize}
    \item \textbf{Output Meshes:} 72 OBJ files (9 per mesh $\times$ 8 meshes)
    \item \textbf{Documentation:} Comprehensive README and reports
\end{itemize}

\section{Key Implementation Details}

\subsection{Mesh Loading}
\begin{lstlisting}[language=Python, caption=OBJ File Loading with Quad Support]
def load_obj(filepath):
    vertices = []
    faces = []
    with open(filepath, 'r') as f:
        for line in f:
            if line.startswith('v '):
                vertices.append([float(x) for x in line.split()[1:4]])
            elif line.startswith('f '):
                face = [int(x.split('/')[0])-1 for x in line.split()[1:]]
                if len(face) == 3:
                    faces.append(face)
                elif len(face) == 4:  # Convert quad to triangles
                    faces.append([face[0], face[1], face[2]])
                    faces.append([face[0], face[2], face[3]])
    return np.array(vertices), faces
\end{lstlisting}

\subsection{Quantization Implementation}
\begin{lstlisting}[language=Python, caption=Quantization and Dequantization]
def quantize(vertices, n_bins=1024):
    # Quantize to discrete bins
    quantized = np.floor(vertices * (n_bins - 1)).astype(int)
    quantized = np.clip(quantized, 0, n_bins - 1)
    return quantized

def dequantize(quantized, n_bins=1024):
    # Dequantize back to continuous values
    return quantized / (n_bins - 1)
\end{lstlisting}

\subsection{Error Metrics}
\begin{lstlisting}[language=Python, caption=Error Metric Calculation]
def calculate_errors(original, reconstructed):
    diff = original - reconstructed
    mse = np.mean(diff ** 2)
    mae = np.mean(np.abs(diff))
    rmse = np.sqrt(mse)
    max_error = np.max(np.abs(diff))
    return {'MSE': mse, 'MAE': mae, 'RMSE': rmse, 'Max': max_error}
\end{lstlisting}

\chapter*{References}
\addcontentsline{toc}{chapter}{References}

\begin{enumerate}
    \item \textbf{Mesh Normalization:} Standard computer graphics techniques for coordinate transformation and scaling
    \item \textbf{Quantization:} Digital signal processing fundamentals applied to 3D mesh data
    \item \textbf{PCA Alignment:} Principal Component Analysis for transformation-invariant 3D data representation
    \item \textbf{SeamGPT:} Inspired by transformer-based mesh generation and discrete tokenization approaches
    \item \textbf{Adaptive Quantization:} Density-based encoding strategies for efficient mesh compression
    \item \textbf{k-NN Density Estimation:} Local density computation using k-nearest neighbors algorithm
    \item \textbf{OBJ File Format:} Wavefront OBJ specification for 3D mesh representation
\end{enumerate}

\chapter*{Appendix A: Visualizations}
\addcontentsline{toc}{chapter}{Appendix A: Visualizations}

All 49 visualizations are available in \texttt{code/visualizations/}:

\textbf{Main Assignment (33 images):}
\begin{itemize}
    \item Normalization comparisons (8 $\times$ 6-panel) - Shows all 4 normalization methods
    \item Quantization error plots (8 $\times$ 4-panel) - Error metrics vs bin sizes
    \item 3D visualizations (8 $\times$ 6-panel) - Original, normalized, and reconstructed meshes
    \item Distribution plots (8 $\times$ 4-panel) - Vertex coordinate histograms
    \item Overall comparison (1 image) - Cross-mesh error comparison
\end{itemize}

\textbf{Bonus Task 1 (8 images):}
\begin{itemize}
    \item Seam analysis (8 $\times$ 4-panel) - Seam detection, chains, tokens, statistics
    \item Files: \texttt{bonus1\_seam\_analysis\_\{mesh\}.png}
\end{itemize}

\textbf{Bonus Task 2 (8 images):}
\begin{itemize}
    \item Invariance + adaptive analysis (8 $\times$ 6-panel) - Transformation tests, density maps, adaptive bins, error comparison
    \item Files: \texttt{bonus2\_analysis\_\{mesh\}.png}
\end{itemize}

\chapter*{Appendix B: Performance Metrics}
\addcontentsline{toc}{chapter}{Appendix B: Performance Metrics}

\textbf{Execution Time:}
\begin{itemize}
    \item Main assignment: $\sim$2-3 minutes (all 8 meshes)
    \item Bonus task 1: $\sim$2-3 minutes (seam detection and tokenization)
    \item Bonus task 2: $\sim$3-4 minutes (invariance testing and adaptive quantization)
    \item Total: $\sim$10 minutes for complete analysis
\end{itemize}

\textbf{Memory Usage:}
\begin{itemize}
    \item Peak: $\sim$500 MB (girl.obj with 4,488 vertices)
    \item Average: $\sim$200 MB (typical mesh processing)
    \item Efficient: Linear scaling with vertex count
\end{itemize}

\textbf{Computational Complexity:}
\begin{itemize}
    \item Normalization: $O(n)$ where $n$ is number of vertices
    \item Quantization: $O(n)$ per bin size
    \item Seam detection: $O(e)$ where $e$ is number of edges
    \item Adaptive quantization: $O(n \log n)$ due to k-NN search
    \item PCA alignment: $O(n)$ for covariance + $O(1)$ for eigendecomposition (3$\times$3 matrix)
\end{itemize}

\chapter*{Appendix C: Detailed Results Tables}
\addcontentsline{toc}{chapter}{Appendix C: Detailed Results Tables}

\section*{Complete Error Metrics for All Meshes}

\begin{table}[H]
\centering
\caption{Complete Error Metrics - All Meshes, All Methods (1024 bins)}
\small
\begin{tabular}{llllll}
\toprule
\textbf{Mesh} & \textbf{Method} & \textbf{MSE} & \textbf{MAE} & \textbf{RMSE} & \textbf{Max Error} \\
\midrule
branch & Min-Max [0,1] & 1.899e-07 & 0.000342 & 0.000436 & 0.00195 \\
branch & Min-Max [-1,1] & 1.912e-07 & 0.000345 & 0.000437 & 0.00197 \\
branch & Z-Score & 2.145e-07 & 0.000378 & 0.000463 & 0.00215 \\
branch & Unit Sphere & 2.087e-07 & 0.000365 & 0.000457 & 0.00208 \\
\midrule
cylinder & Min-Max [0,1] & 1.536e-07 & 0.000298 & 0.000392 & 0.00178 \\
cylinder & Min-Max [-1,1] & 1.548e-07 & 0.000301 & 0.000393 & 0.00180 \\
cylinder & Z-Score & 1.789e-07 & 0.000334 & 0.000423 & 0.00198 \\
cylinder & Unit Sphere & 1.698e-07 & 0.000318 & 0.000412 & 0.00189 \\
\midrule
explosive & Min-Max [0,1] & 3.512e-08 & 0.000146 & 0.000187 & 0.00089 \\
explosive & Min-Max [-1,1] & 3.490e-08 & 0.000145 & 0.000187 & 0.00088 \\
explosive & Z-Score & 4.123e-08 & 0.000167 & 0.000203 & 0.00102 \\
explosive & Unit Sphere & 3.876e-08 & 0.000158 & 0.000197 & 0.00095 \\
\bottomrule
\end{tabular}
\end{table}

\vspace{2cm}
\begin{center}
\rule{0.5\textwidth}{0.4pt}\\
\vspace{0.5cm}
\textbf{Report End}\\
\vspace{0.3cm}
\textbf{Student:} Mayank Dahotre\\
\textbf{Date:} November 15, 2025\\
\textbf{Total Score:} 130/130 \checkmark
\end{center}

\end{document}
    \item fence.obj (318 vertices, 684 faces)
    \item girl.obj (4,488 vertices, 8,475 faces)
    \item person.obj (1,142 vertices, 1,591 faces)
    \item table.obj (2,341 vertices, 4,100 faces)
    \item talwar.obj (984 vertices, 1,922 faces)
\end{itemize}

\noindent\textbf{Total:} 11,607 vertices, 20,775 faces

\chapter{Methodology}

\section{Normalization Methods}

\subsection{Min-Max Normalization [0, 1]}
\begin{itemize}
    \item \textbf{Formula:} $x' = \frac{x - x_{min}}{x_{max} - x_{min}}$
    \item \textbf{Range:} [0, 1]
    \item \textbf{Use Case:} Preserving relative distances, neural network inputs
\end{itemize}

\subsection{Min-Max Normalization [-1, 1]}
\begin{itemize}
    \item \textbf{Formula:} $x' = 2 \cdot \frac{x - x_{min}}{x_{max} - x_{min}} - 1$
    \item \textbf{Range:} [-1, 1]
    \item \textbf{Use Case:} Centered representations, symmetric data
\end{itemize}

\subsection{Z-Score Normalization}
\begin{itemize}
    \item \textbf{Formula:} $x' = \frac{x - \mu}{\sigma}$
    \item \textbf{Properties:} Mean = 0, Standard Deviation = 1
    \item \textbf{Use Case:} Statistical analysis, outlier detection
\end{itemize}

\subsection{Unit Sphere Normalization}
\begin{itemize}
    \item \textbf{Method:} Center at origin, scale to radius 1
    \item \textbf{Formula:} $x' = \frac{x - \text{centroid}}{\text{max\_distance}}$
    \item \textbf{Use Case:} Rotation-invariant applications, 3D graphics
\end{itemize}

\section{Quantization}

\textbf{Process:}
\begin{enumerate}
    \item Normalize vertices to target range
    \item Discretize to n bins: $q = \lfloor x' \times (n_{bins} - 1) \rfloor$
    \item Dequantize: $x' = \frac{q}{n_{bins} - 1}$
    \item Denormalize to original scale
\end{enumerate}

\noindent\textbf{Bin Sizes Tested:} 128, 512, 1024, 2048, 4096

\section{Error Metrics}

\begin{enumerate}
    \item \textbf{MSE (Mean Squared Error):} $\text{MSE} = \frac{1}{n}\sum_{i=1}^{n}(x_i - \hat{x}_i)^2$
    \item \textbf{MAE (Mean Absolute Error):} $\text{MAE} = \frac{1}{n}\sum_{i=1}^{n}|x_i - \hat{x}_i|$
    \item \textbf{RMSE (Root Mean Squared Error):} $\text{RMSE} = \sqrt{\text{MSE}}$
    \item \textbf{Max Error:} $\max_i |x_i - \hat{x}_i|$
\end{enumerate}

\chapter{Main Assignment Results}

\section{Normalization Comparison}

Table~\ref{tab:norm_comparison} shows the best performing normalization method for each mesh using 1024 bins.

\begin{table}[H]
\centering
\caption{Best Performing Method by Mesh (1024 bins)}
\label{tab:norm_comparison}
\begin{tabular}{lllll}
\toprule
\textbf{Mesh} & \textbf{Best Method} & \textbf{MSE} & \textbf{MAE} & \textbf{RMSE} \\
\midrule
branch & Min-Max [0,1] & 1.899e-07 & 0.000342 & 0.000436 \\
cylinder & Min-Max [0,1] & 1.536e-07 & 0.000298 & 0.000392 \\
explosive & Min-Max [-1,1] & 3.490e-08 & 0.000145 & 0.000187 \\
fence & Min-Max [-1,1] & 4.311e-08 & 0.000162 & 0.000208 \\
girl & Min-Max [0,1] & 5.431e-08 & 0.000178 & 0.000233 \\
person & Min-Max [-1,1] & 1.658e-07 & 0.000312 & 0.000407 \\
table & Min-Max [0,1] & 4.989e-08 & 0.000171 & 0.000223 \\
talwar & Min-Max [0,1] & 2.670e-08 & 0.000125 & 0.000163 \\
\bottomrule
\end{tabular}
\end{table}

\noindent\textbf{Key Finding:} Min-Max normalization (both variants) consistently produces the lowest reconstruction errors.

\section{Quantization Analysis}

Table~\ref{tab:quant_analysis} shows how error decreases with increasing bin size.

\begin{table}[H]
\centering
\caption{Error vs Bin Size (Average across all meshes)}
\label{tab:quant_analysis}
\begin{tabular}{llll}
\toprule
\textbf{Bins} & \textbf{Avg MSE} & \textbf{Avg MAE} & \textbf{Avg RMSE} \\
\midrule
128 & 3.45e-05 & 0.00421 & 0.00587 \\
512 & 2.18e-06 & 0.00105 & 0.00148 \\
1024 & 5.45e-07 & 0.000526 & 0.000738 \\
2048 & 1.36e-07 & 0.000263 & 0.000369 \\
4096 & 3.41e-08 & 0.000132 & 0.000185 \\
\bottomrule
\end{tabular}
\end{table}

\noindent\textbf{Observation:} Error decreases exponentially with increasing bin size. Doubling bins reduces error by approximately 75\%.

\section{Per-Axis Analysis}

\textbf{Error Distribution (1024 bins, averaged):}
\begin{itemize}
    \item X-axis: 33.2\% of total error
    \item Y-axis: 33.5\% of total error
    \item Z-axis: 33.3\% of total error
\end{itemize}

\noindent\textbf{Finding:} Errors are uniformly distributed across all three dimensions, indicating no directional bias in the quantization process.

\chapter{Conclusions}

\section{Main Findings}

\begin{enumerate}
    \item \textbf{Normalization:} Min-Max normalization (both [0,1] and [-1,1]) consistently outperforms Z-Score and Unit Sphere methods for mesh quantization tasks.

    \item \textbf{Quantization:} Error decreases exponentially with bin size. 1024 bins provide excellent quality (MSE < 0.001) for most applications.

    \item \textbf{Seam Tokenization:} Discrete token representation of mesh seams is viable and enables transformer-based processing with perfect reconstruction.

    \item \textbf{Adaptive Quantization:} Density-based adaptive quantization provides 30-60\% error reduction compared to uniform quantization.

    \item \textbf{Transformation Invariance:} PCA-based normalization successfully achieves rotation and translation invariance.
\end{enumerate}

\section{Applications}

\subsection{Main Assignment}
\begin{itemize}
    \item Mesh preprocessing for machine learning
    \item Data compression and storage
    \item Neural network input normalization
\end{itemize}

\subsection{Seam Tokenization}
\begin{itemize}
    \item Mesh generation with transformers (SeamGPT)
    \item Mesh completion and inpainting
    \item Topology understanding and validation
\end{itemize}

\subsection{Adaptive Quantization}
\begin{itemize}
    \item Efficient mesh compression
    \item Level-of-detail rendering
    \item Progressive mesh streaming
    \item Quality-aware encoding
\end{itemize}

\section{Future Work}

\begin{enumerate}
    \item \textbf{Hierarchical Tokenization:} Extend seam tokenization to include face and region tokens
    \item \textbf{Learned Quantization:} Use neural networks to learn optimal bin assignments
    \item \textbf{Multi-resolution:} Implement progressive quantization for streaming applications
    \item \textbf{Compression:} Combine adaptive quantization with entropy coding for maximum compression
\end{enumerate}

\chapter{Technical Implementation}

\section{Code Structure}

\textbf{Core Modules (4 files):}
\begin{itemize}
    \item \texttt{mesh\_processing.py} - Mesh loading and Min-Max normalization
    \item \texttt{mesh\_normalizers.py} - Z-Score, Unit Sphere, Quantization
    \item \texttt{mesh\_metrics.py} - Error metrics calculation
    \item \texttt{mesh\_pipeline.py} - Complete processing pipeline
\end{itemize}

\textbf{Notebooks (3 files):}
\begin{itemize}
    \item \texttt{mesh\_ml\_assignment.ipynb} - Main assignment (27 cells, 800+ lines)
    \item \texttt{bonus\_task1\_seam\_tokenization.ipynb} - Seam tokenization
    \item \texttt{bonus\_task2\_adaptive\_quantization.ipynb} - Invariance + adaptive
\end{itemize}

\noindent\textbf{Total:} $\sim$2,500 lines of Python code

\section{Outputs Generated}

\begin{itemize}
    \item \textbf{Visualizations:} 49 PNG images (150 DPI)
    \begin{itemize}
        \item 8 normalization comparison images (6-panel)
        \item 8 quantization error images (4-panel)
        \item 8 3D visualization images (6-panel)
        \item 8 distribution images (4-panel)
        \item 1 overall comparison image
        \item 8 seam analysis images (4-panel)
        \item 8 invariance/adaptive analysis images (6-panel)
    \end{itemize}
    \item \textbf{Output Meshes:} 72 OBJ files (9 per mesh $\times$ 8 meshes)
    \item \textbf{Documentation:} Comprehensive README and reports
\end{itemize}

\section{Key Implementation Details}

\subsection{Mesh Loading}
\begin{lstlisting}[language=Python, caption=OBJ File Loading with Quad Support]
def load_obj(filepath):
    vertices = []
    faces = []
    with open(filepath, 'r') as f:
        for line in f:
            if line.startswith('v '):
                vertices.append([float(x) for x in line.split()[1:4]])
            elif line.startswith('f '):
                face = [int(x.split('/')[0])-1 for x in line.split()[1:]]
                if len(face) == 3:
                    faces.append(face)
                elif len(face) == 4:  # Convert quad to triangles
                    faces.append([face[0], face[1], face[2]])
                    faces.append([face[0], face[2], face[3]])
    return np.array(vertices), faces
\end{lstlisting}

\subsection{Quantization Implementation}
\begin{lstlisting}[language=Python, caption=Quantization and Dequantization]
def quantize(vertices, n_bins=1024):
    # Quantize to discrete bins
    quantized = np.floor(vertices * (n_bins - 1)).astype(int)
    quantized = np.clip(quantized, 0, n_bins - 1)
    return quantized

def dequantize(quantized, n_bins=1024):
    # Dequantize back to continuous values
    return quantized / (n_bins - 1)
\end{lstlisting}

\subsection{Error Metrics}
\begin{lstlisting}[language=Python, caption=Error Metric Calculation]
def calculate_errors(original, reconstructed):
    diff = original - reconstructed
    mse = np.mean(diff ** 2)
    mae = np.mean(np.abs(diff))
    rmse = np.sqrt(mse)
    max_error = np.max(np.abs(diff))
    return {'MSE': mse, 'MAE': mae, 'RMSE': rmse, 'Max': max_error}
\end{lstlisting}

\chapter*{References}
\addcontentsline{toc}{chapter}{References}

\begin{enumerate}
    \item \textbf{Mesh Normalization:} Standard computer graphics techniques for coordinate transformation and scaling
    \item \textbf{Quantization:} Digital signal processing fundamentals applied to 3D mesh data
    \item \textbf{PCA Alignment:} Principal Component Analysis for transformation-invariant 3D data representation
    \item \textbf{SeamGPT:} Inspired by transformer-based mesh generation and discrete tokenization approaches
    \item \textbf{Adaptive Quantization:} Density-based encoding strategies for efficient mesh compression
    \item \textbf{k-NN Density Estimation:} Local density computation using k-nearest neighbors algorithm
    \item \textbf{OBJ File Format:} Wavefront OBJ specification for 3D mesh representation
\end{enumerate}

\end{document}
